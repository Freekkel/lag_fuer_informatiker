\section{Primzahlen}
%
%
%
\subsection{Motivation}
Wie wir gesehen haben, spielt die Bestimmung großer Primzahlen eine wichtige Rolle.
%
%
%
\subsection{\qquad}
  \begin{enumerate}[label={(\alph*)}]
	\item die Verteilung der PRimzahlen in $\mathbb{N}$ ist sehr unregelmäßig. Zu jeder Zahl $s\geq 2$ gibt es $s$ 			
		aufeinanderfolgende Zahlen, die nicht prim sind.
	\begin{description}
		\item [Beweis:] Wählt $t = s +1$ und betrachte $t!+2, t!+3, \dotsc, t!+t$ Offenbar ist $k|t!+k$ für $z \leq k \leq t$
	\end{description}
	\item Die Verteilung der Rpimzahlen in $\mathbb{N}$ ist sehr regelmäßig: Bezeichne mit $\pi(x)$  die Anzahl der Primzahlen 			$\leq x$. Dann nähert sich $\pi(x)$ für wachsende $x$ immer nahe der Funktion $x \rightarrow \frac{x}{\ln(x)}$ an. 		\\
		\qquad\\	
		Genauer: $\lim\limits_{x \to \infty} \frac{x}{\ln(x)} = 1$ (ohne Beweis)
\end{enumerate}
