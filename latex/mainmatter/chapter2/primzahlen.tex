\section{Primzahlen:}
%
%
%
\subsection{Motivation:}
Wie wir gesehen haben, spielt die Bestimmung großer Primzahlen eine wichtige Rolle.
%
%
%
\subsection{\qquad}
  \begin{enumerate}[label={(\alph*)}]
	\item die Verteilung der PRimzahlen in $\mathbb{N}$ ist sehr unregelmäßig. Zu jeder Zahl $s\geq 2$ gibt es $s$ 			
		aufeinanderfolgende Zahlen, die nicht prim sind.
	\begin{description}
		\item [Beweis:] Wähle $t = s +1$ und betrachte $t!+2, t!+3, \dotsc, t!+t$ Offenbar ist $k|t!+k$ für $z \leq k \leq t$
	\end{description}
	\item Die Verteilung der Rpimzahlen in $\mathbb{N}$ ist sehr regelmäßig: Bezeichne mit $\pi(x)$  die Anzahl der Primzahlen 			$\leq x$. Dann nähert sich $\pi(x)$ für wachsende $x$ immer nahe der Funktion $x \rightarrow \frac{x}{\ln(x)}$ an. 		\\
		\qquad\\	
		Genauer: $\lim\limits_{x \to \infty} \frac{x}{\ln(x)} = 1$ (ohne Beweis)
\end{enumerate}
%
%
%
\subsection{Bemerkung:}
Wie viele Primzahlen gibt es zwischen $10^{199}$ und $10^{200}$ ?\\
Wie in 4.2 (b): Ungefähr 
\begin{equation*}
\frac{1}{\ln 10} (\frac{10^{200}}{200} - \frac{10^{199}}{199}) \approx \frac{1}{2,3}\cdot 10^{199}(\frac{1790}{4 \cdot 10^{4}})
\end{equation*}
Die Anzahl der Atome auf der Erde $\approx 10^{51}$ \\
Wir können es gar nicht schaffen, diese Primzahlen alle auszurechnen.
%
%
%
\subsection{Satz (Fermat-Test)}
Genau dann ist $n \geq 2$ eine Primzahl, wenn gilt
\begin{equation*}
	a^{n-1} \equiv_{n} 1 \qquad \forall a \leq \sqrt{n}
\end{equation*}
\begin{description}
	\item[Beweis:] "`$\Rightarrow$ "' Satz von Euler\\
				($n$ Primzahl $\Rightarrow \varphi(n) = n - 1$)\\
				"`$\Leftarrow$"' \\
				Sei $n$ keine Primzahl, etwa $n = a \cdot b$ mit $2 \leq a \leq \sqrt{n}$\\
				Dann $a \ndivides a^{n-1} \Rightarrow n \ndivides a^{n-1} - 1$, d.h. $a^{n-1} \nequiv_{n} 1$
\end{description}
%
%
%
\subsection{Problem:}
Für ein einzelnes $a$ ist die Gleichung $a^{n-1}\equiv_{n} 1$ schnell geprüft:\\
Es dauert jedoch viel zu lang, das für alle $a \leq \sqrt{n}$ zu tun. 
%
%
%
\subsection{Probalistischer Fermat-Test (Ausweg)}
Sei $n \geq 3$ ungerade. Ist $n$ eine Primzahl? 
\begin{description}
	\item[-] Wähle $a \in \{2, \dotsc, n-2\}$ zufällig
	\item[-] Bestimme $d = ggT(a,n)$. Ist $d > 1$, so STOP $\leadsto$ Ausgabe \underline{keine} Primzahl
	\item[-] Andernfalls berechne $a^{n-1}$ mod$n$
	\begin{description}
		\item[-] Ist $a^{n-1} \nequiv_{n} 1$, so STOP $\leadsto$ Ausgabe: \underline{Keine} Primzahl
		\item[-] Ist $a^{n-1} \equiv_{n} 1$, so gehe zurück auf LOS.
	\end{description}
	\item[Idee:]\quad\\
			Entweder stellt sich nach kurzer Zeit heraus, dass $n$ keine Primzahl, oder $n$ ist \underline{mit hoher Wahrscheinlichkeit} eine Primzahl
\end{description}
%
%
%
\subsection{Beispiel:}
\begin{description}
	\item[] Ist $341 = 11 \cdot 31$ eine Primzahl?
	\item[] 1. \underline{Runde} $a = 2$ \quad $ggT(2,341) = 1$ \quad \underline{zeige} $2^{340} \equiv_{341} 1$
	\begin{description}
		\item[] Nun: $1023  = 11 \cdot 93 \Rightarrow 11|2^{10} - 1|2^{340} - 1$ 
		\item[] Ferne: $31|2^{5} - 1|2^{340} - 1$
		\item[] $\Rightarrow 341 = 11 \cdot 31 | 2^{340} - 1$ mit $ggT(11,31) = 11$
		\item[] $\leadsto$ 1. Runde liefert keine Information.
	\end{description}
	\item[] 2. \underline{Runde} $a = 3$ \quad $ggT(341,3) = 1$ \quad Berechne $3^{340}$ mod$341$ 
	\begin{description}
		\item[] $340 = 2^{2} \cdot 85 = 2^{2}(64+16+4+1) = 2^{8}+2^{6}+2^{4}+2^{2}$
	\end{description}
\end{description}

\begin{centering}
	\begin{tabular}{c|c|c|c|c|c|c|c|c} 
		$2^{k}$ & $2^{1}$ &  $2^{2}$ &  $2^{3}$ &  $2^{4}$ &  $2^{5}$ &  $2^{6}$ &  $2^{7}$ &  $2^{8}$  \\\hline 
		$3^{2^{k}}$ mod$341$ & $9$ & $81$ & $6561 \equiv 82$ & $6724 \equiv -96$ & $9216 \equiv 9$ & $81$ & $82$ & $-98$ \\
	\end{tabular}
\end{centering}
\begin{equation*}
	\Rightarrow 3^{340} \equiv_{341} (-96)^{2} \cdot 81^{2} \equiv_{341} 9 \cdot 82 = 738 \nequiv_{341} 1
\end{equation*}
\begin{description}
	\item[Fazit:] $a = 3$ zeigt uns, dass $341$ keine Primzahl ist. Wir nennen $a = 3$ einen Zeugen für $341$. $a = 2$ war kein Zeuge.
	\item[Beachte:] Der Test liefert \underline{keine} Zerlegung von $n = 341$.
\end{description}
%
%
%
\subsection{Bemerkung}
Es gibt Nicht-Primzahlen, für die kein Zeuge existiert. Die kleinste solche ist $561 = 3 \cdot 11 \cdot 17$\\
$n$ Primzahl $\Leftrightarrow a^{n-1} \equiv_{n} 1 \quad \forall 2\leq a \leq n-z$
%
%
%
\subsection{Miller-Rabin-Test}
Sei $n \geq 3$ ungerade. Dann ist $n-1 = 2^{v} \cdot m$ für ein $v \geq 1$ und $m$ ungerade. Es folgt: $a^{n-1} - 1 = (a^{2^{v-1} \cdot m})^{2} - 1^{2} = (a^{2^{v-1} \cdot m} + 1) \cdot (a^{2^{v-1} \cdot m} - 1) = usw. = (a^{2^{v-1} \cdot m} + 1) \cdot (a^{2^{v-2} \cdot m}) \cdot (a^{m} + 1) (a^{m} - 1)$\\
Ist $n$ Primzahl, so muss $n$ eine der Klammern rechts teilen. Wir nennen daher $n$ eine starke Pseudoprimzahl zur Basis $a$, wenn $n$ eine der Klammern teilt. 
\begin{description}
	\item[Klar:] $n$ starke Pseudoprimzahl zu jeder Basis $a \in \{2, \dotsc, n-z\} \Leftrightarrow n$ Primzahl
\end{description}
Beim probabilistischen Miller-Rabin-Test wird in gleicher Weise beim probabilistischen Fermat-Test für diverse Basen geprüft, ob $n$ starke Pseudoprimzahl zur Basis $a$ ist. 
%
%
%
\subsection{Beispiel:}
$n = 561$, $a = 2$, $ggT(a.n) = 1$\\
$2^{560} - 1 = (2^{280} + 1)(2^{240} + 1)(2^{70} + 1)(2^{35} + 1)(2^{35} - 1)$\\
Teste, ob $561$ eine der Klammern teilt.\\
$35 = 32 + 2 + 1 = 2^{5} + 2^{1} + 2^{0}$\\
\quad\\
\begin{centering}
	\begin{tabular}{c|c|c|c|c|c|c}
		$k$ & $0$ & $1$ & $2$ & $3$ & $4$ & $5$\\\hline
		$2^{2^{k}}$ mod$561$ & $2$ & $4$ & $16$ & $256$ & $65536 \equiv -101$ & $10201 \equiv 103$\\
	\end{tabular}
\end{centering}
\quad\\
\quad\\
\begin{centering}
	$\Rightarrow 2^{35} \equiv 103 \cdot 4 \cdot 2 \equiv 8824 \equiv 263 \neq  \pm 1$ mod$561$\\
	$2^{70} \equiv 263^{2} = 69169 \equiv 166 \nequiv \pm 1$ mod$561$\\
	$ 2^{140} \equiv 27556 \equiv 67 \nequiv \pm 1$ mod$561$\\
	$2^{280} \equiv 67^{2} \equiv 4489 \equiv 1 \nequiv - 1$ mod$561$\\
	$\Rightarrow 561$ \underline{keine} Primzahl
\end{centering}
%
%
%
\subsection{Satz:}
Ist $n > 9$ ungerade und keine Primzahl, ist die Anzahl der Basen $a \in \{2, \dotsc, n-z\}$ bzgl. derer $n$ eine starke Pseudoprimzahl ist, $ \leq \frac{\varphi(n)}{4} < \frac{n}{4}$ (ohne Beweis)\\
Somit sind min. $\frac{3}{4}$ aller Basen Zeugen für $n$, und die Wahrscheinlichkeit, dass bei zufällig gewählten $a$ das $n$ starke Pseudoprimzahl ist, ist $< \frac{1}{4}$.\\
Indem wir $20$ Runden durschlaufen, können wir die Wahrscheinlichkeit, dass $n$ immer noch als mögliche Primzahl gehandelt wird, auf $< \frac{1}{4^{20}} \approx \frac{1}{10^{13}}$ senken.\\
Wir können diese Wahrscheinlichkeit unter jede Grenze senken, also z.B. unter die Wahrscheinlichkeit, dass bei der Rechnung ein zufälliger Computerfehler eintritt.
%
%
%