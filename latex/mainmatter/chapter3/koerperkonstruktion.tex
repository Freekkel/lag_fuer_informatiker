\section{Körperkonstruktionen}
Stest sei $K$ ein Körper.
%
%
%
\subsection{Definition und Bemerkung:}
Sei $f \in K[x] \diagdown K$ \underline{fest}. Wir betrachten die Menge $\{g + f \cdot K[x]|g \in K[x]\} = K[x]/f \cdot K[x] = \{g+f\cdot K[x]|g \in K[x]$ mit grad$g <$ grad$f\}$ Wie beim Übergang $\mathbb{Z}\leadsto\mathbb{Z}/n\mathbb{Z}$ wird $K[x]/f \cdot K[x]$ ein kommutativer Rint mit eins vermöge: \\
$\overline{g} + \overline{h} = \overline{g+h}$ und $\overline{g} \cdot \overline{h} = \overline{gh} \ \forall g,h \in K[x]$ wo $ \overline{g} = g + K[x]$\\
Da jede Nebenklasse $\overline{g}$ genau einen Vertreter vom Grad <grad$f$ enthält.\\
Wird also nur mit Polynomen som Grad < grad$f$ gerechnet und bei Überschreiten der Gradgrenze modulo$f$ reduziert. 