\section{Ringe}
%
%
%
\subsection{Definition:}
Ein Ring $R$ sei eine Menge mit Verknüpfungen $+$ und $\cdot$, derart dass gilt:
\begin{enumerate}
	\item $R$ ist eine kommutative Gruppe bzgl. $+$ 
	\begin{description}
		\item[] (neutrales Element: 0)
		\item[] (Inverses zu $a \in R$ bzgl. $+$.:$-a$
	\end{description}
	\item $\cdot$ ist assoziativ: \ $a(bc)=(ab)c \ \forall a,b,c \in R$
	\item Distributivgesetze:
	\begin{description}
		\item[] $a(b+c) = ab + ac \ \forall a,b,c \in R$
		\item[] $(a+b)c = ac + bc \ \forall a,b,c \in R$
	\end{description}
\end{enumerate}
Gibt es ein neutrales Element bzgl. $\cdot$ in $R$, so heißt $R$ \underline{Ring mit Eins}.\\
Ist $\cdot$ kommutativ, so heißt $R$ ein kommutativer Ring. \\
Ist $R\diagdown\{0\}$ bzgl. $\cdot$ eeine kommutative Gruppe, so nennt man $R$ einen Körper.
%
%
%
\subsection{Beispiel:}
\begin{enumerate}[label={(\alph*)}]
	\item $\mathbb{Z}$ und $\mathbb{Z}/n\mathbb{Z}$ sind kommutative Ringe mit Eins.\\
		$\mathbb{Z}/n\mathbb{Z}$ ist Körper $\Leftrightarrow n$ Primzahl (II.2.7)\\
		$\mathbb{Q}$ und $\mathbb{R}$ sind Körper.
	\item Die Menge aller $(n \times n)$-Matrizen mit Einträge aus einem Ring ist ein nicht kommutativer Ring (sofern $n \geq 
		2$ und $R \neq \{0\}$)
	\item In jedem Ring mit Eins bilden die invertierbaren Elemente eine Gruppe.
  \end{enumerate} 
%
%
%
\subsection{Lemma:}
\begin{enumerate}[label={(\alph*)}]
	\item Ist $R$ ein Ring, so gilt stest\\
		$a \cdot 0 = 0 = 0 \cdot a$ und \\
		$(-a) \cdot b = -(ab) = a(-b) \ \forall a,b \in R$
	\item Jeder Körper ist nullteilerfrei, d.h., aus $a \cdot b =0 \Rightarrow a=0 oder b=0.$
  \end{enumerate} 
Beweis:
\begin{enumerate}[label={(\alph*)}]
	\item $a \cdot 0 + a \cdot 0 = a(0+0) = a \cdot 0 = a \cdot 0 + 0$\\ 
		$\mathop{\Rightarrow}\limits^{\mathbb{R}^{+}\text{Gruppe}} a \cdot 0 = 0$ \\
		$ab + a(-b) = a(b+(-b))=a \cdot 0 = 0$\\
		$\Rightarrow a(-b) = -(ab)$
	\item Ist $a \cdot b = 0$ und ist $a \neq 0$, so\\
		$a^{-1}(a \cdot b) = (a^{-1}a)b=1 \cdot b = b$ wobei $a^{-1} = a^{-1} \cdot 0 = 0$
\end{enumerate}
%
%
%
\subsection{Definition:}
 Ringhomomorphismus: $\varphi:R\rightarrow S$\\
Sei ein Gruppenhomomoorphismus bzgl. $+$ mit $\varphi(a \cdot b) = \varphi (a) \cdot \varphi(b) \ \forall a,b \in R$\\
Setze wieder Kern$\varphi = \{a \in R| \varphi(a)=0\}$
%
%
%
\subsection{Bemerkung:}
In gleicher Weise wie bei Gruppen gilt dann:
\begin{enumerate}[label={(\alph*)}]
	\item Bild $\varphi$ ist ein Unterring von $S$.
	\item Kern$\varphi$ ist ein sogenanntes Ideal von $R$, d.h., es gilt:
	\begin{description}
		\item[-] Kern$\varphi$ ist Untergruppe von $R^{+}$
		\item[-] magnetische Eigenschaft: $a \cdot r \in$ Kern$\varphi \wedge r \cdot a \in$ Kern$\varphi \ \forall a \in$ 
			Kern$\varphi$ und $r \in R$
		\item[-] die Fasern von $\varphi$ sind genau die Mengen (Kern$\varphi$)$+a$ $(a \in R)$\\
				Wir haben also wieder eine Bijektion
		\begin{equation*}
			(Kern\varphi) + a \mathop{\longleftrightarrow}\limits^{\tilde{\varphi}} \varphi(a)
		\end{equation*}
		$R\diagdown$Kern$\varphi = \{$ Kern$\varphi +a| a \in R\} \longleftrightarrow$ Bild$\varphi$\\
		Wieder kann $R\diagdown$Kern$\varphi$ vermöge $\tilde{\varphi}$ zu einem Ring gemacht werden.\\
		$I :=$ Kern$\varphi$
		\begin{equation*}
			(I+a)+(I+b)=I+(a+b)
		\end{equation*}
		\begin{equation*}
			(I+a)\cdot(I+b)=I+(ab) \ \forall a,b \in R
		\end{equation*}
	\end{description}
\end{enumerate}
